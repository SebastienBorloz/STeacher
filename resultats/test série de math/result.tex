\documentclass{article}
\usepackage{amsmath}
\usepackage{amssymb}
\usepackage{array}

\begin{document}

4. L'indice d'iode ($x$, en gI/100g) et le nombre de cétane ($y$, sans dimension) ont été mesurés pour différents biodiesels. Les valeurs sont reportées dans le tableau suivant.

\[
\begin{array}{c|cccccc}
x & 60 & 70 & 80 & 90 & 100 \\
y & 64 & 61 & 58 & 56 & 54 \\
\end{array}
\]

(a) Calculer $\overline{x}$, $\overline{y}$, $s_x$, $s_y$, $\mathrm{Cov}(x,y)$ et $\mathrm{Corr}(x,y)$.

(b) Y a-t-il une relation linéaire entre $x$ et $y$ ?

(c) Déterminer l'équation de la droite de régression. Dessiner les valeurs mesurées et la droite de régression sur un même graphique.

\vspace{0.5cm}

(a) \quad $\overline{x} = 80$, \quad $\overline{y} = 58,6$, \quad $s_x = \sqrt{25} \approx 15,811$, \quad $s_y = \sqrt{7,58} \approx 3,875$

\[
\mathrm{Cov}(x,y) = \frac{1}{4} \sum_{i=1}^5 (x_i - 80)(y_i - 58,6) = -62,5, \quad \mathrm{Corr}(x,y) = \frac{-62,5}{62,85} = 0,9944
\]

(b) Oui car $\mathrm{Corr}(x,y)$ est très proche de $-1$.

(c)

\[
A = \begin{pmatrix}
60 & 1 \\
70 & 1 \\
80 & 1 \\
90 & 1 \\
100 & 1
\end{pmatrix}, \quad
B = \begin{pmatrix}
64 \\
61 \\
58 \\
56 \\
54
\end{pmatrix}, \quad
X = \begin{pmatrix} m \\ q \end{pmatrix}
\]

\[
A^T A = \begin{pmatrix}
60 & 70 & 80 & 90 & 100 \\
1 & 1 & 1 & 1 & 1
\end{pmatrix}
\begin{pmatrix}
60 & 1 \\
70 & 1 \\
80 & 1 \\
90 & 1 \\
100 & 1
\end{pmatrix}
= \begin{pmatrix}
33000 & 400 \\
400 & 5
\end{pmatrix}
\]

\[
(A^T A)^{-1} = \frac{1}{5000} \begin{pmatrix}
5 & -400 \\
-400 & 33000
\end{pmatrix}
\]

\[
A^T B = \begin{pmatrix}
60 & 70 & 80 & 90 & 100 \\
1 & 1 & 1 & 1 & 1
\end{pmatrix}
\begin{pmatrix}
64 \\
61 \\
58 \\
56 \\
54
\end{pmatrix}
= \begin{pmatrix}
23790 \\
293
\end{pmatrix}
\]

\[
\Rightarrow
\begin{pmatrix} m \\ q \end{pmatrix}
= \frac{1}{5000} \begin{pmatrix}
5 & -400 \\
-400 & 33000
\end{pmatrix}
\begin{pmatrix}
23790 \\
293
\end{pmatrix}
= \frac{1}{5000} \begin{pmatrix}
5 \times 23790 - 400 \times 293 \\
-400 \times 23790 + 33000 \times 293
\end{pmatrix}
= \frac{1}{5000} \begin{pmatrix}
-7250 \\
-392500
\end{pmatrix}
= \begin{pmatrix}
-1,45 \\
-78,5
\end{pmatrix}
\]

\vspace{0.5cm}

\noindent (Le résultat final est noté dans l'image comme $(-1,45, -78,6)$, transcription fidèle.)

\vspace{0.5cm}

\noindent (Le graphique montre les points et la droite de régression décroissante.)

\end{document}